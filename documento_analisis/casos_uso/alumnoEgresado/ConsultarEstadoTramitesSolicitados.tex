\begin{UseCase}{CUEE03}{Consultar estatus de los trámites solicitados}{El alumno o egresado podrá consultar el estado en el que se encuentra cada documento que haya solicitado.}
	\UCitem{Versión}{1.0}
	\UCitem{Autor}{Estudiante/Egresado.}	
	\UCitem{Propósito}{El estudiante o egresado podrá visualizar el estado de los tramites solicitados.}
	\UCitem{Entradas}{
		\begin{itemize}
			\item Identificador: Cadena de caracteres la cual servirá para identificar el usuario.
		\end{itemize}			
	
	}
	\UCitem{Salidas}{
	\begin{itemize}
		\item Mensaje de notificación:  "No hay documentos solicitados".
\end{itemize}		
	}
	
	
	\UCitem{Precondiciones}{
		\begin{itemize}
			\item El alumno o egresado debió ser identificado por el sistema.
			\item Debe existir al menos una solicitud de un documento.
		\end{itemize}
	
	
	}
	\UCitem{Tipo de ejecución}{Secundario, viene de CUG01 Iniciar sesión.}
	\UCitem{Volatilidad}{Baja.}
	\UCitem{Madurez}{Baja.}
	\UCitem{Prioridad}{Alta.}
	\UCitem{Estado}{En proceso}
	\UCitem{Autor}{Chávez Chávez Javier}
	\UCitem{Revisor}{-}
	
\end{UseCase}

\begin{UCtrayectoria}{Principal}
	\UCpaso [\UCactor] Da click el botón ''Estatus de los documentos'' ubicado en la pantalla Inicio.
	\UCpaso Verifica si existen documentos solicitados.[TAA]
	\UCpaso Obtiene los datos de los documentos.
	\UCpaso Muestra en pantalla el estatus de los documentos.
\end{UCtrayectoria}

\begin{UCtrayectoriaA}{A}{No existen documentos}
	\UCpaso Muestra el mensaje "No hay documentos solicitados'' en la pantalla de Estatus de los documentos.
\end{UCtrayectoriaA}
