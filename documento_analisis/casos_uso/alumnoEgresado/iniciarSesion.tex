\begin{UseCase}{CUG01}{Iniciar sesión  Estudiante/Egresado.}{Método para autentificar un usuario ante el sistema.}
	\UCitem{Versión}{1.0}
  
  \UCitem{Actor}{Estudiante/Egresado}
  
  \UCitem{Propósito}{Evitar el acceso al sistema de personas ajenas a la institución, así como el robo de identidad con el fin de solicitar trámites innecesarios.}
  
  \UCitem{Entradas}{
    \begin{itemize}
      \item Boleta: Cadena conformada por 10 números, iniciada por el año de ingreso del alumno, seguida de 63 , código perteneciente a la institución y finalmente su número de estudiante asignado en formato de 4 dígitos.
      \item Contraseña: Campo de tipo password.
    \end{itemize}
  }
  \UCitem{Origen}{
    \begin{itemize}
      \item Desde teclado:  Todas las entradas.
      \end{itemize}
  }
  \UCitem{Salidas}{
    \begin{itemize}
      \item Mensaje de notificación  {\bf AlertaConfirmarInicioSesión}``Inicio de sesión existoso".
    \end{itemize}
  }
  \UCitem{Destino}{
    \begin{itemize}
      \item Mensaje de notificación : Pantalla.
    \end{itemize}
  }
  \UCitem{Precondiciones}{
   \begin{itemize}
      \item Que el estudiante no haya solicitado baja temporal o baja definitiva.
      \item Que exista un registro del estudiante en el repositorio de datos.
    \end{itemize}
  }
  \UCitem{Postcondiciones}{
    \begin{itemize}
      \item El alumno queda autentificado ante el sistema.
    \end{itemize}
  }
  \UCitem{Observaciones}{
    \begin{itemize} \item Para la primera vez de uso del sistema, el estudiante ingresará su CURP en la sección de contraseña, posteriormente podrá modificar dicha sección.
  	\end{itemize}
  }
  \UCitem{Errores}{
  	\begin{itemize} \item Ninguno
  	\end{itemize}
	}
  \UCitem{Tipo de ejecución}{Primaria}
	\UCitem{Volatilidad}{Alta}
	\UCitem{Madurez}{Media}
	\UCitem{Prioridad}{Alta}
  \UCitem{Estado}{Autorizado}
	\UCitem{Autor}{Marcela Castro Flores}
	\UCitem{Revisor}{Alberto Maldonado Romo}
\end{UseCase}

\begin{UCtrayectoria}{Principal}
  \UCpaso[\UCactor] Solicita ingresar al sistema
  \UCpaso Muestra la pantalla UIniciarSesión.
  \UCpaso[\UCactor] Ingresa su número de boleta y su contraseña.
  \UCpaso[\UCactor]Da click en el botón \IUbutton{Iniciar Sesión}.
  \UCpaso Valida el formato de la entrada de los datos. [TAA]
  \UCpaso Obtiene los datos y verifica si se encuentran en el repositorio de datos. [TAB]
  \UCpaso Muestra el mensaje {\bf AlertaConfirmaciónIniciarSesión}.
  \UCpaso[\UCactor]Ingresa al sistema.
\end{UCtrayectoria}
-9

\begin{UCtrayectoriaA}{A}{Datos ingresados incorrectos}
  \UCpaso Notifica que el formato de la boleta es incorrecta y muestra mensaje {\bf MFormatoIncorrecto1} ``Ingresa únicamente números".
  \UCpaso Continúa transacción desde el paso 5.
\end{UCtrayectoriaA}

\begin{UCtrayectoriaA}{B}{Datos no encontrados en el repositorio de datos}
  \UCpaso Notifica que los datos ingresados son incorrectos o no existen y muestra alerta {\bf AlertaDatosIncorrectos}.
  \UCpaso Continúa transacción desde el paso 2.
\end{UCtrayectoriaA}
%-------------------------------------- TERMINA descripción del caso de uso.