\begin{UseCase}{CUEE02}{Solicitar trámite}{Un alumno o egresado requiere hacer una petición de trámite a control escolar}
	\UCitem{Version}{0.1}
  \UCitem{Actor}{Alumno/Egresado}
  \UCitem{Proposito}{El paciente evita ir físicamente a las oficinas de control escolar y tiene la posibilidad de pedir el tipo de documento deseado via internet.}
  \UCitem{Entradas}{
    \begin{itemize}
    \item Tipo de documento a pedir: Cadena de carácteres  predeterminada que significará el tipo de documento  que el alumno solicita.
    \item Razón de petición: Cadena de carácteres predeterminada que sirve para dar una explicación de por qué se está pidiendo el documento.
    \end{itemize}
  }
  \UCitem{Origen}{
    \begin{itemize}
    \item Html select: Todos.
      \end{itemize}
  }
  \UCitem{Salidas}{
    \begin{itemize}
		\item Mensaje de confirmación {\bf MSG1a-}``Estás seguro de solicitar este documento? [Nombre del documento pedido]".
		\item Mensaje de información {\bf MSG2a-}``Lo siento, has excedido el número de peticiones  disponibles. "
    \end{itemize}
  }
  \UCitem{Destino}{
    \begin{itemize}
      \item MSG1a,MSG2a, : Pantalla.
    \end{itemize}
  }
  \UCitem{Precondiciones}{
   \begin{itemize}
      \item El alumno debió haber sido identificado por el sistema.
    \end{itemize}
  }
  \UCitem{Postcondiciones}{
    \begin{itemize}
      \item Se registra la petición del alumno y/o egresado en el sistema.
    \end{itemize}
  }
  \UCitem{Observaciones}{
		\begin{itemize}
			\item El alumno puede pedir un máximo de 5 constancias al mes.
		\end{itemize}
  }
  \UCitem{Errores}{
  	\begin{itemize}
  	\item Ninguno.
  	\end{itemize}
	}
  \UCitem{Tipo de ejecución}{Secundaria, viene de CU1 Iniciar Sesión}
	\UCitem{Volatilidad}{Baja}
	\UCitem{Madurez}{Alta}
	\UCitem{Prioridad}{Alta}
  \UCitem{Estado}{En proceso}
	\UCitem{Autor}{Rubén Murga Dionicio}
	\UCitem{Revisor}{-----}
\end{UseCase}

\begin{UCtrayectoria}{Principal}
  \UCpaso[\UCactor] solicita un trámite haciendo click en la opción del menú `` Solicitar trámite”.
  \UCpaso El sistema muestra la pantalla UIEESolicitarTramite
  \UCpaso[\UCactor] indica el tipo de documento deseada haciendo click sobre una opción del select tipo de documento a pedir  que se mencionó en las entradas previamente.
  \UCpaso[\UCactor] Indica el la razón de la petición haciendo click sobre una opción del select de razón de petición que se mencionó en las entradas previamente.
  \UCpaso[\UCactor] El actor da click en el botón\IUbutton{Solicitar trámite} 
  \UCpaso  Válida que el número de peticiones en el mes sea menor a 5 peticiones.[TAA]
  \UCpaso El sistema almacena la petición en el repositorio de datos.
\end{UCtrayectoria}

\begin{UCtrayectoriaA}{A}{Peticiones excedidas} 
  \UCpaso El sistema inhabilita el botón  	\IUbutton{Solicitar trámite} 
  \UCpaso continúa transacción desde el paso 5.
\end{UCtrayectoriaA}

%\begin{UCtrayectoriaA}{B}{Solicitudes excedidas.}
  %\UCpaso notifica que ya hay una cita registrada con los mismos datos, mostrando el mensaje {\bf MSG2b-} “Este horario fue agendado, seleccione otro por favor”.
 % \UCpaso regresa a la vista UI2A Agendar Citas.
  %\UCpaso continúa transacción desde el paso 5.
%\end{UCtrayectoriaA}
%-------------------------------------- TERMINA descripción del caso de uso.