\begin{UseCase}{CUEE02}{Solicitar trámite}{Un alumno o egresado requiere hacer una petición de trámite a control escolar}
	\UCitem{Version}{0.1}
  \UCitem{Actor}{Alumno/Egresado}
  \UCitem{Proposito}{El paciente evita ir físicamente a las oficinas de control escolar y tiene la posibilidad de pedir el tipo de documento deseado via internet.}
  \UCitem{Entradas}{
    \begin{itemize}
    \item Tipo de documento a pedir: Cadena de carácteres  predeterminada que significará el tipo de documento  que el alumno solicita.
    \item Razón de petición: Cadena de carácteres predeterminada que sirve para dar una explicación de por qué se está pidiendo el documento.
    \end{itemize}
  }
  \UCitem{Origen}{
    \begin{itemize}
    \item Seleccionable: Tipo de documento a pedir. Razón de petición
      \end{itemize}
  }
  \UCitem{Salidas}{
    \begin{itemize}
		\item Mensaje de confirmación {\bf AlertaConfirmarCorreo-}`` Tu solicitud se enviará a control escolar y te enviaremos un correo cuando haya sido aceptada y cuando esté lista para recogerse.  
	[cuerpoCorreo@servidor.dominio]
	¿Es correcto tu correo? Actualiza en caso de que no sea correcto	".
		\item Mensaje de información {\bf AlertaLimitePeticion}``El número máximo de peticiones es 5."
		\item Mensaje de información {\bf AlertasolicitudEnviada} ``Se ha mandado tu solicitud, puedes verificar en qué estapa se encuentra en el apartado de "Mis solicitudes en proceso"
    \end{itemize}
  }
  \UCitem{Destino}{
    \begin{itemize}
      \item AlertaConfirmarCorreo,AlertaLimitePeticion,AlertasolicitudEnviada : Pantalla.
    \end{itemize}
  }
  \UCitem{Precondiciones}{
   \begin{itemize}
      \item El alumno debió haber sido identificado por el sistema.
    \end{itemize}
  }
  \UCitem{Postcondiciones}{
    \begin{itemize}
      \item Se registra la petición del alumno y/o egresado en el sistema.
    \end{itemize}
  }
  \UCitem{Observaciones}{
		\begin{itemize}
			\item El alumno puede pedir un máximo de 5 constancias al mes.
			\item Las opciones de documentos son: 
			\begin{itemize}
				\item Constancia 
				\item Boleta
			\end{itemize}
			\item Las opciones de razones de petición de trámite son: 
			\begin{itemize}
				\item Actividad Cultural
				\item Actividad Deportida
			\end{itemize}
		\end{itemize}
  }
  \UCitem{Errores}{
  	\begin{itemize}
  	\item Ninguno.
  	\end{itemize}
	}
  \UCitem{Tipo de ejecución}{Secundaria, viene de CU1 Iniciar Sesión}
	\UCitem{Volatilidad}{Baja}
	\UCitem{Madurez}{Alta}
	\UCitem{Prioridad}{Alta}
  \UCitem{Estado}{En proceso}
	\UCitem{Autor}{Rubén Murga Dionicio}
	\UCitem{Revisor}{-----}
\end{UseCase}

\begin{UCtrayectoria}{Principal}
  \UCpaso[\UCactor] solicita un trámite haciendo click en la opción del menú `` Solicitar trámite”.
  \UCpaso El sistema muestra la pantalla  {\bf UIEESolicitarTramite}
  \UCpaso[\UCactor] indica el tipo de documento deseada haciendo click sobre una opción del select tipo de documento a pedir  que se mencionó en las entradas previamente.
  \UCpaso[\UCactor] Indica el la razón de la petición haciendo click sobre una opción del select de razón de petición que se mencionó en las entradas previamente.
  \UCpaso[\UCactor] El actor da click en el botón\IUbutton{Agregar a lista de peticiones} 
  \UCpaso  Válida que el número de peticiones en el mes sea menor a 5 peticiones.[TAA] 
  \UCpaso Agrega el tipo de documento junto con su razón de pedido a la lista de peticiones [TAB]
  \UCpaso[\UCactor] El actor da click en el botón\IUbutton{Enviar lista de peticiones} 
  \UCpaso Muestra la alerta {\bf AlertaConfirmarCorreo} 
  \UCpaso[\UCactor] El actor verifica que su correo sea correcto. [TAC]
  \UCpaso Almacena la petición en el repositorio de datos.
  \UCpaso Muestra la alerta  {\bf AlertasolicitudEnviada}
\end{UCtrayectoria}

\begin{UCtrayectoriaA}{A}{Peticiones excedidas} 
\UCpaso Muestra el mensaje {\bf AlertaLimitePeticion}
  \UCpaso Inhabilita el botón  	\IUbutton{Solicitar trámite} 
  \UCpaso continúa transacción desde el paso 8.
\end{UCtrayectoriaA}

\begin{UCtrayectoriaA}{B}{Agregar documento a lista} 
  \UCpaso Regresa a trayectoria principal paso 3.
\end{UCtrayectoriaA}

\begin{UCtrayectoriaA}{C}{Actualizar correo electrónico} 
  \UCpaso[\UCactor] Escribe un nuevo correo electrónico 
  \UCpaso[\UCactor] Da click en el botón \IUbutton{Estoy seguro, enviar lista de peticiones.} [TAD]
  \UCpaso Actualiza en el repositorio de datos el correo del alumno o egresado.
  \UCpaso Esconde la alerta {\bf AlertaConfirmarCorreo} 
\end{UCtrayectoriaA}

\begin{UCtrayectoriaA}{D}{No enviar lista de peticiones} 
 \UCpaso[\UCactor] Da click en el botón \IUbutton{No, no enviar lista de peticiones.}
 \UCpaso Esconde la alerta  {\bf AlertaConfirmarCorreo} 
 \UCpaso Muestra pantalla anterior (UIEESolicitarTramite) con la lista de peticiones llenada anteriormente.
\end{UCtrayectoriaA}

%-------------------------------------- TERMINA descripción del caso de uso.