\begin{UseCase}{CUG01}{Solicitar trámite}{Método para autentificar un usuario ante el sistema.}
	\UCitem{Versión}{0.1}
  \UCitem{Actor}{Estudiante/Egresado - Analista - Jefa de Gestión Escolar - Departamentos escolares - Autoridad pertinente}
  \UCitem{Propósito}{Evitar el acceso al sistema de personas ajenas a la institución, así como el robo de identidad con el fin de solicitar trámites innecesarios.}
  \UCitem{Entradas}{
    \begin{itemize}
      \item : cadena aleatoria de caracteres formada por MongoDB
      \item Fecha de cita: cadena de caracteres con el formato ``DD/MM/YYYY HH:MM'', donde DD, MM, YYYY, HH, MM son número enteros que representan el día, mes, año, hora y minutos de la fecha respectivamente.
      \item Consultorio: Número entero positivo entre 0 y 12, representa el número de consultorio.
      \item Médico:  Nombre del médico que esta asignado al consultorio en ese horario de cita.
    \end{itemize}
  }
  \UCitem{Origen}{
    \begin{itemize}
<<<<<<< HEAD
      \item Id de la cita:  Variable de sesión de la aplicación.
      \item Fecha de cita: Desde un calendario ordenado por mes que cambia dinamicamente para seleccionar el  año, mes, día, hora y minutos .
      \item Consultorio: Desde una lista.
      \item Médico: de un cuadro de texto de sólo lectura, se obtiene mediante una consulta.
=======
    \item Seleccionable: Tipo de documento a pedir. Razón de petición
>>>>>>> d8ab1e6361cf6b16c2bf87a5e11ae5d266619ba9
      \end{itemize}
  }
  \UCitem{Salidas}{
    \begin{itemize}
<<<<<<< HEAD
      \item Mensaje de notificación  {\bf MSG2a-}``Registro de cita exitoso".
      \item Mensaje de notificación  {\bf MSG2b-}``Este horario fue agendado, seleccione otro por favor".
       \item Mensaje de notificación  {\bf MSG2c-}``Seleccione un horario".

=======
		\item Mensaje de confirmación {\bf AlertaConfirmarCorreo-}`` Tu solicitud se enviará a control escolar y te enviaremos un correo cuando haya sido aceptada y cuando esté lista para recogerse.  
	[cuerpoCorreo@servidor.dominio]
	¿Es correcto tu correo? Actualiza en caso de que no sea correcto	".
		\item Mensaje de información {\bf AlertaLimitePeticion}``El número máximo de peticiones es 5."
		\item Mensaje de información {\bf AlertasolicitudEnviada}
>>>>>>> d8ab1e6361cf6b16c2bf87a5e11ae5d266619ba9
    \end{itemize}
  }
  \UCitem{Destino}{
    \begin{itemize}
<<<<<<< HEAD
      \item Mensaje de notificación : Pantalla.
=======
      \item AlertaConfirmarCorreo,AlertaLimitePeticion,AlertasolicitudEnviada : Pantalla.
>>>>>>> d8ab1e6361cf6b16c2bf87a5e11ae5d266619ba9
    \end{itemize}
  }
  \UCitem{Precondiciones}{
   \begin{itemize}
      \item Que haya un médico asignado en la misma fecha, hora y en el consultorio seleccionado.
      \item Que no haya una cita registrada en la misma fecha, hora y en el consultorio seleccionado.
    \end{itemize}
  }
  \UCitem{Postcondiciones}{
    \begin{itemize}
      \item Se registra la cita del actor en el sistema.
      \item La cita registrada aparece en la lista de citas registradas del paciente.
    \end{itemize}
  }
  \UCitem{Observaciones}{
<<<<<<< HEAD
    Las citas agendadas se bloquean del calendario.
=======
		\begin{itemize}
			\item El alumno puede pedir un máximo de 5 constancias al mes.
			\item Las opciones de documentos son: 
			\begin{itemize}
				\item Constancia 
				\item Boleta
			\end{itemize}
			\item Las opciones de razones de petición de trámite son: 
			\begin{itemize}
				\item 
			\end{itemize}
		\end{itemize}
>>>>>>> d8ab1e6361cf6b16c2bf87a5e11ae5d266619ba9
  }
  \UCitem{Errores}{
  	El actor selecciona una fecha, hora y consultorio en los cuales ya existe una cita previamente registrada, se informa al actor mostrando el mensaje {\bf MSG2b-} ``Este horario fue agendado, seleccione otro por favor'' y se continúa desde el paso 5 
	}
  \UCitem{Tipo de ejecución}{Secundaria, viene de CU1 Iniciar Sesión}
	\UCitem{Volatilidad}{Media}
	\UCitem{Madurez}{Media}
	\UCitem{Prioridad}{Alta}
  \UCitem{Estado}{En revisión}
	\UCitem{Autor}{Adrián Eduardo Galindo García}
	\UCitem{Revisor}{Rubén Murga Dionicio}
\end{UseCase}

\begin{UCtrayectoria}{Principal}
<<<<<<< HEAD
  \UCpaso[\UCactor] solicita agendar cita haciendo clic en la opción del menú “Agendar cita”.
  \UCpaso obtiene el total de citas disponibles para cada día.
  \UCpaso muestra el calendario de la UI2A Agendar Citas con la cantidad de citas disponibles por día.
  \UCpaso deshabilita los días del calendario que hayan llegado al máximo de citas disponibles.
  \UCpaso[\UCactor] indica la fecha de la cita deseada haciendo click sobre el día.
  \UCpaso oculta el calendario, y muestra un listado de horas disponibles para el día seleccionado. UI2B Agendar Citas
  \UCpaso[\UCactor] indica la hora de la cita deseada haciendo click sobre ella.
   \UCpaso oculta el listado de horas, y muestra un listado de horas con minutos, disponibles para la hora seleccionada. UI2C Agendar Citas
   \UCpaso[\UCactor] indica la hora con minutos de la cita deseada haciendo click sobre ella.
  \UCpaso habilita la selección del listado de los consultorios.
  \UCpaso[\UCactor] indica el consultorio deseado seleccionandolo de la lista. [Trayectoria A]
  \UCpaso obtiene el nombre del médico asignado al consultorio en la fecha y hora seleccionadas. 
  \UCpaso muestra los datos seleccionados de la cita y habilita el botón \IUbutton{Agendar cita}.
  \UCpaso[\UCactor] da click en el botón \IUbutton{Agendar cita}.
  \UCpaso verifica que no se haya registrado una cita con los mismos datos [Trayectoria B].
  \UCpaso notifica el resultado de la operación mostrando la información de la cita.
\end{UCtrayectoria}

\begin{UCtrayectoriaA}{A}{Horario no seleccionado}
  \UCpaso notifica que no ha seleccionado un horario de cita, mostrando el mensaje {\bf MSG2c-} “Seleccione un horario”.
  \UCpaso regresa a la vista UI2A Agendar Citas.
  \UCpaso continúa transacción desde el paso 5.
\end{UCtrayectoriaA}

\begin{UCtrayectoriaA}{B}{Confirmación de la creación de la cita}
  \UCpaso notifica que ya hay una cita registrada con los mismos datos, mostrando el mensaje {\bf MSG2b-} “Este horario fue agendado, seleccione otro por favor”.
  \UCpaso regresa a la vista UI2A Agendar Citas.
  \UCpaso continúa transacción desde el paso 5.
\end{UCtrayectoriaA}
=======
  \UCpaso[\UCactor] solicita un trámite haciendo click en la opción del menú `` Solicitar trámite”.
  \UCpaso El sistema muestra la pantalla  {\bf UIEESolicitarTramite}
  \UCpaso[\UCactor] indica el tipo de documento deseada haciendo click sobre una opción del select tipo de documento a pedir  que se mencionó en las entradas previamente.
  \UCpaso[\UCactor] Indica el la razón de la petición haciendo click sobre una opción del select de razón de petición que se mencionó en las entradas previamente.
  \UCpaso[\UCactor] El actor da click en el botón\IUbutton{Agregar a lista de peticiones} 
  \UCpaso  Válida que el número de peticiones en el mes sea menor a 5 peticiones.[TAA] 
  \UCpaso Agrega el tipo de documento junto con su razón de pedido a la lista de peticiones [TAB]
  \UCpaso[\UCactor] El actor da click en el botón\IUbutton{Enviar lista de peticiones} 
  \UCpaso Muestra la alerta {\bf AlertaConfirmarCorreo} 
  \UCpaso[\UCactor] El actor verifica que su correo sea correcto. [TAC]
  \UCpaso Almacena la petición en el repositorio de datos.
  \UCpaso Muestra la alerta  {\bf AlertasolicitudEnviada}
\end{UCtrayectoria}

\begin{UCtrayectoriaA}{A}{Peticiones excedidas} 
\UCpaso Muestra el mensaje {\bf AlertaLimitePeticion}
  \UCpaso Inhabilita el botón  	\IUbutton{Solicitar trámite} 
  \UCpaso continúa transacción desde el paso 8.
\end{UCtrayectoriaA}

\begin{UCtrayectoriaA}{B}{Agregar documento a lista} 
  \UCpaso Regresa a trayectoria principal paso 3.
\end{UCtrayectoriaA}

\begin{UCtrayectoriaA}{C}{Actualizar correo electrónico} 
  \UCpaso[\UCactor] Escribe un nuevo correo electrónico 
  \UCpaso[\UCactor] Da click en el botón \IUbutton{Estoy seguro, enviar lista de peticiones.} [TAD]
  \UCpaso Actualiza en el repositorio de datos el correo del alumno o egresado.
  \UCpaso Esconde la alerta {\bf AlertaConfirmarCorreo} 
\end{UCtrayectoriaA}

\begin{UCtrayectoriaA}{D}{No enviar lista de peticiones} 
 \UCpaso[\UCactor] Da click en el botón \IUbutton{No, no enviar lista de peticiones.}
 \UCpaso Esconde la alerta  {\bf AlertaConfirmarCorreo} 
 \UCpaso Muestra pantalla anterior (UIEESolicitarTramite) con la lista de peticiones llenada anteriormente.
\end{UCtrayectoriaA}

>>>>>>> d8ab1e6361cf6b16c2bf87a5e11ae5d266619ba9
%-------------------------------------- TERMINA descripción del caso de uso.