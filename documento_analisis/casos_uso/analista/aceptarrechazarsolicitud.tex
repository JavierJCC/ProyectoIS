\pagebreak
\begin{UseCase}{CUA03.1}{Aceptar o rechazar solicitudes}{Un analista 
visualizará las solicitudes de trámites que se han generado, y así poder 
aceptar o cancelarla dicha solicitud.}
	\UCitem{Versión}{1.0}
  \UCitem{Actor}{Analista}
  \UCitem{Propósito}{El analista podrá aceptar o rechazar las solicitudes 
realizadas por un estudiante, exestudiante(egresado o estudiante dado de 
baja, ya sea temporal o definitiva) y/o personal de alguna área de la 
escuela, dependiendo de la situación que se encuentre el usuario al cual se 
le va a realizar el documento.}
  \UCitem{Entradas}{
    \begin{itemize}
    \item El analista debió haber sido identificado por el sistema.
		\item El analista debe oprimir un botón ya sea, aceptar o rechazar.
    \end{itemize}
  }
	\UCitem{Origen}{
		\begin{itemize}
			\item Caso de uso CUA03.1 Gestionar solicitudes
		\end{itemize}
	}
  \UCitem{Salidas}{
    \begin{itemize}
		\item Mensaje de solicitud {\bf AlertaSolicitaMotivo-}. ``Indique el 
motivo por el cual se rechaza la solicitud.''
		\item Mensaje de confirmacion {\bf AlertaConfirmarRechazar-}. ``La 
solicitud ha sido rechazada.''
    \end{itemize}
		\item Mensaje de solicitud {\bf AlertaSolicitaFolio-}. ``Ingrese el folio del documento que fue asignado''
		\item Mensaje de confirmacion {\bf AlertaConfirmarAceptacion-}. ``La 
solicitud ha aceptada.''
    \end{itemize}
		\item Mensaje de confirmacion {\bf AlertaError-}. ``No se realizó ningún cambio, intente más tarde.''
    \end{itemize}
  }
  \UCitem{Destino}{
    \begin{itemize}
      \item AlertaSolicitaMotivo, AlertaConfirmarRechazar, AlertaSolicitaFolio, AlertaConfirmaAceptacion, AlertaError: Pantalla.
    \end{itemize}
  }
  \UCitem{Precondiciones}{
   \begin{itemize}
      \item Un estudiante, egresado o departamento escolar debió haber 
solicitado algún trámite.
    \end{itemize}
  }
  
  \UCitem{Observaciones}{
		\begin{itemize}
			\item El analista en dado de que se equivoque odrá cambiar de aceptar a rechazar o viceversa en la opcion que se encuentra del lado izquierdo en el menu.
			\end{itemize}
		\end{itemize}
  }
  \UCitem{Errores}{
  	\begin{itemize}
  	\item Ninguno.
  	\end{itemize}
	}
  \UCitem{Tipo de ejecución}{Secundaria, viene de CUG03 Gestionar solicitudes}
	\UCitem{Volatilidad}{Baja}
	\UCitem{Madurez}{Alta}
	\UCitem{Prioridad}{Alta}
  \UCitem{Estado}{Autorizado}
	\UCitem{Autor}{Juan Antonio Guzmán Chávez}
	\UCitem{Revisor}{Rubén Murga Dionicio}
\end{UseCase}

\begin{UCtrayectoria}{Principal}
  \UCpaso[\UCactor] Visualiza las solicitudes haciendo clic en la opción del menú ''Gestionar trámite'' {\bf CUA03Gestionarsolicitudes}.
  \UCpaso [\UCactor] Oprime el botón de aceptar solicitud\IUbutton{Aceptar}[TAA].
	\UCpaso Muestra un modal en la pantalla {\bf AlertaSolicitaFolio}.
  \UCpaso [\UCactor] Da click en el botón \IUbutton{Aceptar}.[TAB]
	\UCpaso Realiza los cambios en la base de datos.[TAC]
	\UCpaso Muestra una alerta en pantalla {\bf AlertaConfirmaAceptacion}.
	\UCpaso Muestra la pantalla {\bf CUA03Gestionarsolicitudes}.
\end{UCtrayectoria}

\begin{UCtrayectoriaA}{A}{Oprime el boton Rechazar solicitud} 
	\UCpaso Muestra un modal en la pantalla {\bf AlertaSolicitaMotivo}.
  \UCpaso [\UCactor] Da click en el botón \IUbutton{Aceptar}.[TAB]
	\UCpaso Continua transacción en el paso 5.
\end{UCtrayectoriaA}

\begin{UCtrayectoriaA}{B}{Oprime boton cancelar} 
  \UCpaso Cierra el modal.
  \UCpaso Continua transacción en el paso 1.
\end{UCtrayectoriaA}

\begin{UCtrayectoriaA}{C}{No se realizan los cambios en la base} 
  \UCpaso  Muestra una alerta en pantalla {\bf AlertaError}.
  \UCpaso Continua transacción en el paso 1.
\end{UCtrayectoriaA}

%-------------------------------------- TERMINA descripción del caso de uso.