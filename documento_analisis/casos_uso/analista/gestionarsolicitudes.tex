\pagebreak
\begin{UseCase}{CUA03}{Gestionar solicitudes}{Un analista requiere visualizar las solicitudes de trámites que se han generado.}
	\UCitem{Versión}{1.0}
  \UCitem{Actor}{Analista}
  \UCitem{Propósito}{El analista podrá visualizar las solicitudes de los estudiantes, egresados o departamentos escolares que estén pendientes o que aún no estén autorizadas.}
  \UCitem{Origen}{
    \begin{itemize}
    \item Repositorio de datos.
    \end{itemize}
  }
  \UCitem{Salidas}{
    \begin{itemize}
		\item Tabla de información: Visualización de solicitudes hechas por estudiantes, exestudiantes (egresados o alumnos dados de baja, ya sea temporal o definitiva) o áreas de la escuela.
    \end{itemize}
  }
  \UCitem{Destino}{
    \begin{itemize}
      \item Pantalla.
    \end{itemize}
  }
  \UCitem{Precondiciones}{
   \begin{itemize}
      \item Un estudiante, egresado o departamento escolar debió haber solicitado algún trámite de documento.
    \end{itemize}
  }
  
  \UCitem{Observaciones}{
		\begin{itemize}
			\item El solicitante puede ser: 
			\begin{itemize}
				\item Estudiante
				\item Egresado
				\item Departamentos escolares
			\end{itemize}
		\end{itemize}
  }
  \UCitem{Errores}{
  	\begin{itemize}
  	\item Ninguno.
  	\end{itemize}
	}
  \UCitem{Tipo de ejecución}{Secundaria, viene de CUG02 Iniciar Sesión Empleado}
	\UCitem{Volatilidad}{Media}
	\UCitem{Madurez}{Baja}
	\UCitem{Prioridad}{Alta}
  \UCitem{Estado}{Autorizado}
	\UCitem{Autor}{Juan Antonio Guzmán Chávez}
	\UCitem{Revisor}{Rubén Murga Dionicio}
\end{UseCase}

\begin{UCtrayectoria}{Principal}
  \UCpaso[\UCactor] Visualiza las solicitudes haciendo clic en la opción del menú ''Gestionar trámite''.
  \UCpaso Realiza la consulta y obtiene los datos del repositorio de datos. [TAA]
  \UCpaso Muestra la pantalla con las solicitudes {\bf CUA03Gestionarsolicitudes}.[TAC]
\end{UCtrayectoria}

\begin{UCtrayectoriaA}{A}{Error al cargar datos} 
\UCpaso Muestra la alerta {\bf AlertaErrorDeConexión}.
  \UCpaso[\UCactor] Da click en el botón \IUbutton{Volver a cargar datos}. [TAB]
  \UCpaso Continua transacción desde el paso 2.
\end{UCtrayectoriaA}

\begin{UCtrayectoriaA}{B}{No Recarga los datos} 
  \UCpaso Da click en el botón \IUbutton{Cancelar}.
  \UCpaso Continua transacción en el paso 1.
\end{UCtrayectoriaA}

\begin{UCtrayectoriaA}{C}{No hay solicitudes pendientes} 
  \UCpaso Muestra la pantalla {\bf No hay solicitudes pendientes}.
\end{UCtrayectoriaA}

%-------------------------------------- TERMINA descripción del caso de uso.
