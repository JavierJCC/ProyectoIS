\begin{UseCase}{CUG01}{Iniciar sesión}{Método para autentificar un usuario ante el sistema.}
	\UCitem{Versión}{0.1}
  \UCitem{Actor}{Estudiante/Egresado - Analista - Jefa de Gestión Escolar - Departamentos escolares - Autoridad pertinente}
  \UCitem{Propósito}{Evitar el acceso al sistema de personas ajenas a la institución, así como el robo de identidad con el fin de solicitar trámites innecesarios.}
  \UCitem{Entradas}{
    \begin{itemize}
      \item Boleta: cadena formada por 10 números, iniciada por el año de ingreso del alumno, seguida de 63 , código perteneciente a la institución y finalmente su número de estudiante asignado en formato de 4 dígitos.
      \item CURP: cadena de caracteres con el formato- primeras 2 letras del apellido paterno, primera letra del apellido materno y primera letra del nombre,``YY/MM/DD'', donde YY, MM, DD, son número enteros que representan el año,día y mes de nacimiento de la persona,``H o M'' donde H significa sexo hombre y M mujer y primeras dos letras del estado de nacimiento seguido de 3 letras y 2 números que conforman la clave homogénea que es irrepetible y es asignada aleatoriamente a cada persona.
    \end{itemize}
  }
  \UCitem{Origen}{
    \begin{itemize}
      \item Desde teclado:  Todas las entradas.
      \end{itemize}
  }
  \UCitem{Salidas}{
    \begin{itemize}
      \item Mensaje de notificación  {\bf AlertaConfirmarInicioSesión}``Inicio de sesión existoso".
    \end{itemize}
  }
  \UCitem{Destino}{
    \begin{itemize}
      \item Mensaje de notificación : Pantalla.
    \end{itemize}
  }
  \UCitem{Precondiciones}{
   \begin{itemize}
      \item Que el estudiante no haya solicitado baja temporal o baja definitiva.
      \item Que exista un registro del estudiante en el repositorio de datos.
    \end{itemize}
  }
  \UCitem{Postcondiciones}{
    \begin{itemize}
      \item El alumno queda autentificado ante el sistema.
    \end{itemize}
  }
  \UCitem{Observaciones}{
    \begin{itemize}Ninguno
  	\end{itemize}
  }
  \UCitem{Errores}{
  	\begin{itemize}Ninguno
  	\end{itemize}
	}
  \UCitem{Tipo de ejecución}{Primaria}
	\UCitem{Volatilidad}{Alta}
	\UCitem{Madurez}{Media}
	\UCitem{Prioridad}{Alta}
  \UCitem{Estado}{Esperando revisión}
	\UCitem{Autor}{Marcela Castro Flores}
	\UCitem{Revisor}{}
\end{UseCase}

\begin{UCtrayectoria}{Principal}
  \UCpaso[\UCactor] Solicita ingresar al sistema
  \UCpaso Muestra la pantalla UIniciarSesión.
  \UCpaso[\UCactor] Ingresa su número de boleta y su CURP.
  \UCpaso[\UCactor]Da click en el botón \IUbutton{Iniciar Sesión}.
  \UCpaso Valida el formato de la entrada de los datos [TAA].
  \UCpaso Obtiene los datos y verifica si se encuentran en el repositorio de datos [TAB].
  \UCpaso Muestra el mensaje AlertaConfirmaciónIniciarSesión. 
  \UCpaso[\UCactor]Ingresa al sistema.
\end{UCtrayectoria}
-9

\begin{UCtrayectoriaA}{A}{Datos ingresados incorrectos}
  \UCpaso Notifica que el formato de la boleta es incorrecta y muestra mensaje{\bf MFormatoIncorrecto1} ``Debe ingresar únicamente números".
  \UCpaso Notifica que el formato de CURP es incorrecto y muestra mensaje{\bf MFormatoIncorrecto2} ``Debe ingresar el formato XXXX000000XXXXXX00".
  \UCpaso Continúa transacción desde el paso 5.
\end{UCtrayectoriaA}

\begin{UCtrayectoriaA}{B}{Datos no encontrados en el repositorio de datos}
  \UCpaso Notifica que los datos ingresados son incorrectos o no existen y muestra alerta {\bf AlertaDatosIncorrectos}.
  \UCpaso Continúa transacción desde el paso 2.
\end{UCtrayectoriaA}
%-------------------------------------- TERMINA descripción del caso de uso.