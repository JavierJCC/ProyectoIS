
\begin{UseCase}{CUG02}{Iniciar sesión empleado}{Método para autentificar un usuario ante el sistema.}
	\UCitem{Versión}{1.0}
  \UCitem{Actor}{Analista - Jefa de Gestión Escolar - Departamentos escolares - Autoridad pertinente}
  \UCitem{Propósito}{
	\begin{itemize}
	  \item Evitar el acceso al sistema de personas ajenas a la institución.
	\end{itemize}   
 }
  \UCitem{Entradas}{
    \begin{itemize}
      \item No. de empleado: cadena conformada por 8 números.
      \item Contraseña: Campo de tipo password.
    \end{itemize}
  }
  \UCitem{Origen}{
    \begin{itemize}
      \item Desde teclado:  Todas las entradas.
      \end{itemize}
  }
  \UCitem{Salidas}{
    \begin{itemize}
      \item Mensaje de notificación  {\bf AlertaConfirmarInicioSesión}``Inicio de sesión existoso".
    \end{itemize}
  }
  \UCitem{Destino}{
    \begin{itemize}
      \item Mensaje de notificación : Pantalla.
    \end{itemize}
  }
  \UCitem{Precondiciones}{
   \begin{itemize}
      \item Que el actor se encuentre laborando actualmente en la institución.
      \item Que exista un registro del actor en el repositorio de datos.
    \end{itemize}
  }
  \UCitem{Postcondiciones}{
    \begin{itemize}
      \item El actor queda autentificado ante el sistema.
    \end{itemize}
  }
  \UCitem{Observaciones}{
    \begin{itemize} 
    	\item Ninguno
  	\end{itemize}
  }
  \UCitem{Errores}{
  	\begin{itemize} 
  		\item Ninguno
  	\end{itemize}
	}
  \UCitem{Tipo de ejecución}{Primaria}
	\UCitem{Volatilidad}{Alta}
	\UCitem{Madurez}{Media}
	\UCitem{Prioridad}{Alta}
  \UCitem{Estado}{Esperando revisión}
	\UCitem{Autor}{Marcela Castro Flores}
	\UCitem{Revisor}{}
\end{UseCase}

\begin{UCtrayectoria}{Principal}
  \UCpaso[\UCactor] Solicita ingresar al sistema.
  \UCpaso Muestra la pantalla UIniciarSesión.
  \UCpaso[\UCactor] Ingresa su número de no. de empleado y su RFC.
  \UCpaso[\UCactor]Da click en el botón \IUbutton{Iniciar Sesión}.
  \UCpaso Valida el formato de la entrada de los datos [TAA].
  \UCpaso Obtiene los datos y verifica si se encuentran en el repositorio de datos [TAB].
  \UCpaso Muestra el mensaje {\bf AlertaConfirmaciónIniciarSesión. }
  \UCpaso[\UCactor]Ingresa al sistema.
\end{UCtrayectoria}


\begin{UCtrayectoriaA}{A}{Datos ingresados incorrectos}
  \UCpaso Notifica que el formato de la boleta es incorrecta y muestra mensaje{\bf MFormatoIncorrecto1} ``Ingresa únicamente números".
  \UCpaso Notifica que el formato de CURP es incorrecto y muestra mensaje {\bf MFormatoIncorrecto2} ``Ingresa el formato solicitado como en la muestra".
  \UCpaso Continúa transacción desde el paso 5.
\end{UCtrayectoriaA}

\begin{UCtrayectoriaA}{B}{Datos no encontrados en el repositorio de datos}
  \UCpaso Notifica que los datos ingresados son incorrectos o no existen y muestra alerta {\bf AlertaDatosIncorrectos}.
  \UCpaso Continúa transacción desde el paso 2.
\end{UCtrayectoriaA}
%-------------------------------------- TERMINA descripción del caso de uso.