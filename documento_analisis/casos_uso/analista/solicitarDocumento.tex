\pagebreak
\begin{UseCase}{CUG04}{Solicitar documento}{Un alumno o egresado requiere hacer una petición de trámite a control escolar}
	\UCitem{Version}{1.0}
  \UCitem{Actor}{Analista}
  \UCitem{Propósito}{El actor registra una petición de documento.}
  \UCitem{Entradas}{
    \begin{itemize}
    \item Documento: Cadena de carácteres  predeterminada que significará el tipo de documento  que el alumno solicita.
    \item Motivo de petición del documento: Cadena de carácteres predeterminada que sirve para dar una explicación de por qué se está pidiendo el documento.
    \end{itemize}
    \item Boleta: Cadena de carácteres de longitud 10 que contiene el año en el que entró el alumno, un código de escuela y números 4 números que se asignan al alumno
  }
  \UCitem{Origen}{
    \begin{itemize}
    \item Seleccionable: Documento, Motivo de petición del documento
    \item Campo de texto: Boleta
      \end{itemize}
  }
  \UCitem{Salidas}{
    \begin{itemize}
		\item Mensaje de confirmación {\bf AlertaConfirmarCorreo-}`` Le enviaremos un correo al alumno/egresado cuando haya sido aceptada y cuando esté lista para recogerse.  
	[cuerpoCorreo@servidor.dominio]
	¿Es correcto tu correo? Actualiza en caso de que no sea correcto	".
		\item Mensaje de información {\bf AlertaLimitePeticion}``El número máximo de peticiones es 5."
		\item Mensaje de información {\bf AlertasolicitudEnviada} ``Se ha mandado la solicitud, El alumno puede verificar en qué estapa se encuentra en el apartado de "Mis solicitudes en proceso." que se le mostrará cuando acceda a su cuenta."
		
		\item Mensaje de información {\bf AlertaValidaNumDocumentos} ``Debe haber al menos un documento en tu petición."
		
		\item Mensaje de información {\bf AlertaValidaMotivo} ``Debe introducir un motivo válido para la petición."
		
		\item Mensaje de información {\bf AlertaSolicitudCancelada} ``No se ha enviado la solicitud, puede seguir agregando o eliminando documentos de la lista de peticiones."
		
    \end{itemize}
  }
  \UCitem{Destino}{
    \begin{itemize}
      \item AlertaConfirmarCorreo,AlertaLimitePeticion,AlertasolicitudEnviada,AlertaValidaMotivo,AlertaValidaNumDocumentos : Pantalla.
    \end{itemize}
  }
  \UCitem{Precondiciones}{
   \begin{itemize}
      \item El actor debió haber sido identificado por el sistema.
      \item El alumno debe estar inscrito.
    \end{itemize}
  }
  \UCitem{Postcondiciones}{
    \begin{itemize}
      \item Se registra la petición del alumno y/o egresado en el sistema.
    \end{itemize}
  }
  \UCitem{Observaciones}{
		\begin{itemize}
			\item El actor puede pedir un máximo de 5 constancias al mes.
			\item Las opciones de documentos son: 
			\begin{itemize}
				\item Constancia 
				\item Boleta
			\end{itemize}
			\item Las opciones de razones de petición de trámite son: 
			\begin{itemize}
				\item Actividad Cultural
				\item Actividad Deportiva
			\end{itemize}
			\item El actor puede agregar más de un documento.
		\end{itemize}
  }
  \UCitem{Errores}{
  	\begin{itemize}
  	\item Ninguno.
  	\end{itemize}
	}
  \UCitem{Tipo de ejecución}{Secundaria, viene de CU01 Iniciar Sesión}
	\UCitem{Volatilidad}{Baja}
	\UCitem{Madurez}{Alta}
	\UCitem{Prioridad}{Alta}
  \UCitem{Estado}{Autorizado}
	\UCitem{Autor}{Rubén Murga Dionicio}
	\UCitem{Revisor}{Javier Chávez Chávez}
\end{UseCase}

\begin{UCtrayectoria}{Principal}
  \UCpaso[\UCactor] Solicita un documento haciendo click en la opción del menú `` Solicitar documento".
  \UCpaso Muestra la pantalla  {\bf UIEESolicitarTramite}.
  \UCpaso[\UCactor] Indica el tipo de documento deseada seleccionandolo del seleccionable de {\bf Documento } 
  \UCpaso[\UCactor] Indica el motivo de petición seleccionandolo del seleccionable {\bf Motivo de petición del documento}
  \UCpaso[\UCactor] Da click en el botón\IUbutton{ Agregar a lista de documentos}. 
  \UCpaso  Valida que el número de peticiones en el mes sea menor a 5 peticiones. [TAA] 
  \UCpaso Valida que el actor haya seleccionado un motivo válido. [TAD]
  \UCpaso Agrega el tipo de documento junto con su razón de pedido a la lista de peticiones mostrada en pantalla.
  \UCpaso[\UCactor]  Da click en el botón\IUbutton{Enviar lista de documentos}.
  \UCpaso Muestra la alerta {\bf AlertaConfirmarCorreo} 
  \UCpaso[\UCactor] Verifica que su correo sea correcto. [TAB]
  \UCpaso[\UCactor] Da click en el botón \IUbutton{Enviar}. [TAC]
  \UCpaso Actualiza el correo electrónico que el actor escribió anteriormente.
  \UCpaso Almacena en el repositorio de datos los documentos que quedaron en la lista.
  \UCpaso Muestra la alerta  {\bf AlertasolicitudEnviada}
\end{UCtrayectoria}

\begin{UCtrayectoriaA}{A}{Peticiones excedidas} 
\UCpaso Muestra el mensaje {\bf AlertaLimitePetición}.
  \UCpaso No agrega el documento a la lista de documentos.
\end{UCtrayectoriaA}

\begin{UCtrayectoriaA}{B}{Correo electrónico incorrecto.} 
  \UCpaso[\UCactor] Escribe un nuevo correo electrónico 
\end{UCtrayectoriaA}

\begin{UCtrayectoriaA}{C}{Solicitud cancelada} 
 \UCpaso Esconde la alerta  {\bf AlertaConfirmarCorreo}.
 \UCpaso Muestra la alerta {\bf AlertaSolicitudCancelada }
 \UCpaso Muestra pantalla anterior {\bf (UIEESolicitarTramite)} con la lista de peticiones llenada anteriormente.
\end{UCtrayectoriaA}

\begin{UCtrayectoriaA}{D}{Motivo inválido} 
\UCpaso Muestra el mensaje {\bf AlertaValidaMotivo}.
  \UCpaso No agrega el documento a la lista de documentos.
\end{UCtrayectoriaA}

%-------------------------------------- TERMINA descripción del caso de uso.