\pagebreak
\begin{UseCase}{CUEE02}{Solicitar documento}{Un alumno o egresado requiere hacer una petición de trámite a control escolar}
	\UCitem{Version}{1.0}
  \UCitem{Actor}{Alumno/Egresado}
  \UCitem{Propósito}{El actor evita ir físicamente a las oficinas de control escolar y tiene la posibilidad de pedir el tipo de documento deseado via internet.}
  \UCitem{Entradas}{
    \begin{itemize}
    \item Documento: Cadena de carácteres  predeterminada que significará el tipo de documento  que el alumno solicita.
    \item Motivo de petición del documento: Cadena de carácteres predeterminada que sirve para dar una explicación de por qué se está pidiendo el documento.
    \end{itemize}
  }
  \UCitem{Origen}{
    \begin{itemize}
    \item Seleccionable: Documento. Motivo de petición del documento
      \end{itemize}
  }
  \UCitem{Salidas}{
    \begin{itemize}
		\item Mensaje de confirmación {\bf AlertaConfirmarCorreo-}`` Tu solicitud se enviará a control escolar y te enviaremos un correo cuando haya sido aceptada y cuando esté lista para recogerse.  
	[cuerpoCorreo@servidor.dominio]
	¿Es correcto tu correo? Actualiza en caso de que no sea correcto	".
		\item Mensaje de información {\bf AlertaLimitePeticion}``El número máximo de peticiones es 5."
		\item Mensaje de información {\bf AlertasolicitudEnviada} ``Se ha mandado tu solicitud, puedes verificar en qué estapa se encuentra en el apartado de "Mis solicitudes en proceso.""
		
		\item Mensaje de información {\bf AlertaValidaNumDocumentos} ``Debes tener al menos un documento en tu petición."
		
		\item Mensaje de información {\bf AlertaValidaMotivo} ``Debes introducir un motivo válido para tu petición."
		
		\item Mensaje de información {\bf AlertaSolicitudCancelada} ``No se ha enviado tu solicitud, puedes seguir agregando o eliminando documentos de tu lista."
		\item Mensaje de información {\bf AlertaEgresado} ``Sólo puedes solicitar boleta global."
		
    \end{itemize}
  }
  \UCitem{Destino}{
    \begin{itemize}
      \item AlertaConfirmarCorreo : Pantalla
      \item AlertaLimitePeticion:Pantalla
      \item AlertaSolicitudEnviada:pantalla
      \item AlertaValidaMotivo :Pantalla
      \item AlertaValidaNumDocumentos : Pantalla
      \item AlertaEgresado: Pantalla
    \end{itemize}
  }
  \UCitem{Precondiciones}{
   \begin{itemize}
      \item El actor debió haber sido identificado por el sistema.
    \end{itemize}
  }
  \UCitem{Postcondiciones}{
    \begin{itemize}
      \item Se registra la petición del alumno y/o egresado en el sistema.
    \end{itemize}
  }
  \UCitem{Observaciones}{
		\begin{itemize}
			\item El actor puede pedir un máximo de 5 constancias al mes.
			\item Las opciones de documentos son: 
			\begin{itemize}
				\item  Boleta global
				\item Boleta certificada
				\item Boleta departamental
				\item Constancia de inscripción
				\item Constancia de estudios
				\item Constancia con periodo vacacional
				\item Constancia para trámite de SS
				\item Constancia de prácticas profesionales
				\item Constancia de inscripción y horario
			\end{itemize}
			\item Las opciones de razones de petición de trámite son: 
			\begin{itemize}
				\item Actividad Cultural
				\item Actividad Deportiva
			\end{itemize}
			\item El actor puede agregar más de un documento.
		\end{itemize}
  }
  \UCitem{Errores}{
  	\begin{itemize}
  	\item Ninguno.
  	\end{itemize}
	}
  \UCitem{Tipo de ejecución}{Secundaria, viene de CU01 Iniciar Sesión Estudiante/Empleado}
	\UCitem{Volatilidad}{Baja}
	\UCitem{Madurez}{Alta}
	\UCitem{Prioridad}{Alta}
  \UCitem{Estado}{Autorizado}
	\UCitem{Autor}{Rubén Murga Dionicio}
	\UCitem{Revisor}{Javier Chávez Chávez}
\end{UseCase}

\begin{UCtrayectoria}{Principal}
  \UCpaso[\UCactor] Solicita un documento haciendo click en la opción del menú `` Solicitar documento".
  \UCpaso Muestra la pantalla  {\bf UIEESolicitarTramite}.
  \UCpaso[\UCactor] Indica el tipo de documento deseada seleccionandolo del seleccionable de {\bf Documento } 
  \UCpaso[\UCactor] Indica el motivo de petición seleccionandolo del seleccionable {\bf Motivo de petición del documento}
  \UCpaso[\UCactor] Da click en el botón\IUbutton{ Agregar a lista de documentos}. 
  \UCpaso  Valida que el número de peticiones en el mes sea menor a 5 peticiones. [TAA] 
  \UCpaso Valida que el actor haya seleccionado un motivo válido. [TAD]
  \UCpaso Valida si el actor es alumno y valida los documentos disponibles a solicitar [TAE]
  \UCpaso Agrega el tipo de documento junto con su razón de pedido a la lista de peticiones mostrada en pantalla.
  \UCpaso[\UCactor]  Da click en el botón\IUbutton{Enviar lista de documentos}.
  \UCpaso Muestra la alerta {\bf AlertaConfirmarCorreo} 
  \UCpaso[\UCactor] Verifica que su correo sea correcto. [TAB]
  \UCpaso[\UCactor] Da click en el botón \IUbutton{Enviar}. [TAC]
  \UCpaso Actualiza el correo electrónico que el actor escribió anteriormente.
  \UCpaso Almacena en el repositorio de datos los documentos que quedaron en la lista.
  \UCpaso Muestra la alerta  {\bf AlertasolicitudEnviada}
\end{UCtrayectoria}

\begin{UCtrayectoriaA}{A}{Peticiones excedidas} 
\UCpaso Muestra el mensaje {\bf AlertaLimitePetición}.
  \UCpaso No agrega el documento a la lista de documentos.
\end{UCtrayectoriaA}

\begin{UCtrayectoriaA}{B}{Correo electrónico incorrecto.} 
  \UCpaso[\UCactor] Escribe un nuevo correo electrónico 
\end{UCtrayectoriaA}

\begin{UCtrayectoriaA}{C}{Solicitud cancelada} 
 \UCpaso Esconde la alerta  {\bf AlertaConfirmarCorreo}.
 \UCpaso Muestra la alerta {\bf AlertaSolicitudCancelada }
 \UCpaso Muestra pantalla anterior {\bf (UIEESolicitarTramite)} con la lista de peticiones llenada anteriormente.
\end{UCtrayectoriaA}

\begin{UCtrayectoriaA}{D}{Motivo inválido} 
\UCpaso Muestra el mensaje {\bf AlertaValidaMotivo}.
  \UCpaso No agrega el documento a la lista de documentos.
\end{UCtrayectoriaA}


\begin{UCtrayectoriaA}{E}{No alumno} 
\UCpaso Muestra el mensaje {\bf AlertaEgresado}.
  \UCpaso No agrega el documento a la lista de documentos.
\end{UCtrayectoriaA}
%-------------------------------------- TERMINA descripción del caso de uso.