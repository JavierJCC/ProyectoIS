\pagebreak
\begin{UseCase}{CUDES01}{Enviar memorandum}{Un usuario de un departamento escolar manda un memorandum y una lista de alumnos a control escolar.}
	\UCitem{Version}{1.0}
  \UCitem{Actor}{Departamento escolar}
  \UCitem{Propósito}{El actor da a conocer una petición de constancia o boleta a control escolar.}
  \UCitem{Entradas}{
    \begin{itemize}
    \item Documento: Archivo con terminación ``.pdf",``.docx", ``.xlsx". de cualquier tamaño.
    \end{itemize}
  }
  \UCitem{Origen}{
    \begin{itemize}
    \item input tipo file: Documento.
      \end{itemize}
  }
  \UCitem{Salidas}{
    \begin{itemize}
		\item Mensaje de confirmación {\bf DocumentoAgregado-}`` Se ha cargado el archivo [nombre del archivo]."
		\item Mensaje de información {\bf DocumentoNoAgregado}``No se ha podido cargar el archivo. Sube un archivo con extensión valida.."
    \end{itemize}
  }
  \UCitem{Destino}{
    \begin{itemize}
      \item DocumentoAgregado,DocumentoNoAgregado: Pantalla.
    \end{itemize}
  }
  \UCitem{Precondiciones}{
   \begin{itemize}
      \item El actor debió haber sido identificado por el sistema como un actor perteneciente a un departamento escolar.
    \end{itemize}
  }
  \UCitem{Postcondiciones}{
    \begin{itemize}
      \item Se carga o se cargan uno o múltiples documentos al sistema.
    \end{itemize}
  }
  \UCitem{Observaciones}{
		\begin{itemize}
			\item Las extensiones permitidas son las siguientes:
				\begin{itemize}
					\item docx
					\item pdf
					\item xlsx
				\end{itemize}
				\item Es posible agregar más de un documento.
		\end{itemize}
  }
  \UCitem{Errores}{
  	\begin{itemize}
  	\item DocumentoNoAgregado.
  	\end{itemize}
	}
  \UCitem{Tipo de ejecución}{Secundaria, viene de CU02 Iniciar Sesión}
	\UCitem{Volatilidad}{Baja}
	\UCitem{Madurez}{Media}
	\UCitem{Prioridad}{Alta}
  \UCitem{Estado}{En proceso}
	\UCitem{Autor}{Rubén Murga Dionicio}
	\UCitem{Revisor}{Javier Chávez Chávez}
\end{UseCase}

\begin{UCtrayectoria}{Principal}
  \UCpaso[\UCactor] Solicita un documento haciendo click en la opción del menú `` Agregar memorandum".
  \UCpaso Muestra la pantalla  {\bf UIDEMemorandum}.
  \UCpaso[\UCactor] Da click en el botón \IUbutton{Elegir un archivo}
  \UCpaso Valida que la extensión del archivo se encuentre en una de las extensiones válidas mencionadas previamente en el apartado de observaciones. [TAA]
  \UCpaso Añade el documento al servidor.
  \UCpaso Registra el nuevo documento al repositorio de datos. 
  \UCpaso muestra la alerta {\bf DocumentoAgregado}

\end{UCtrayectoria}

\begin{UCtrayectoriaA}{A}{Extensión invalida} 
\UCpaso Muestra el mensaje {\bf DocumentoNoAgregado}.
  \UCpaso No agrega el documento a la lista de documentos.
\end{UCtrayectoriaA}


%-------------------------------------- TERMINA descripción del caso de uso.